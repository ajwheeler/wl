\documentclass{article}

\usepackage{amssymb}
\usepackage{amsmath}
\usepackage{amsthm}
\usepackage{enumerate}
\usepackage{graphicx}
\usepackage[colorlinks]{hyperref}

\usepackage{margins}
\usepackage{adam}
%---------------------------------------------------------------------
\begin{document}
%---------------------------------------------------------------------
\begin{abstract}
 A number of large current and forthcoming astronomical surveys (for
 instance DES, LSST, WFIRST) plan to use the spatially coherent
 distortion of galaxy ellipticities by gravitational lensing to map
 the distribution of dark matter and energy in the universe.  Here we
 explore for the first time \comment{?} the possibility of using
 physically motivated models of galaxies to improve shape measurement.
 We use Markov-chain Monte Carlo in conjunction with state-of-the-art
 lensing simiulation software (GalSim) to show that well-understood
 scaling relations between observable properties of galaxies can be
 used to remove intrinsic alignment contamination.  Specifically, we
 investigate statistical power of the Kormendy Relation, the color
 difference between disks and bulges, and the connection between the
 3-D orientation of galactic disks and their measured shapes for the
 purposes of galaxy shape measurement.
\comment{
Once the simulations are complete, we will use software (Balrog) and
statistical methods that have been developed by this group for similar
purposes to deploy our new intrinsic alignment mitigation on Dark
Energy Survey data.
}
\end{abstract}

\section{The Concept}
We use a simple disk + bulge galaxy model (draw by galsim) to model
the effects of weak lensing.  We would like to see how much simple
empirical priors improve out fits.

\section{The Model}
The model is a simple disk+bulge galaxy, with the following parameters:
\begin{itemize}
\item $R_{disk}$, disk half-light radius
\item $F_{disk}$, disk flux
\item $g_1^{disk}$, disk intrinsic alignment
\item $g_2^{disk}$, disk intrinsic alignment
\item $R_{bulge}$, bulge half-light radius
\item $F_{bulge}$, bulge flux
\item $g_1^{bulge}$, bulge intrinsic alignment
\item $g_2^{bulge}$, bulge intrinsic alignment
\red{
\item $g_1^{shear}$, lensing shear
\item $g_2^{shear}$, lensing shear
\item $mu$, lensing magnification
}
\end{itemize}
$g_1^{disk}$, $g_2^{disk}$, $g_1^{bulge}$, and $g_2^{shear}$ all model
intrinsic alignment, but are parameterized like reduced shear in weak
lensing formalism because it's easy to apply with galsim.  The bulge
is drawn with a De Vaucouleurs profile, and the disk is drawn with an
$n=1$ S\'{e}rsic profile.  The image is convolved with a $\sigma = 0.25
\textrm{ arcsecond}$ Gaussian PSF.

\subsection{The color ``prior''}
One important aspect of the model is that the disk and bulge can
optionally be drawn with different $r-g$ colors such that
$(g-r)_{disk} \approx (g-r)_{bulge} - 2$. \comment{n.b. I have to check to
ensure that these magnitudes are being calculated correctly.}  We have
been treating the color difference between the disk and bulge like a
prior in that we are interested in how much switching it on affects
our fits, but our current treatment presumes perfect knowledge of the
color difference between disk and bulge.


\section{Priors}
Besides the ``prior'' on disk and bulge color, we are applying two 
empirically motivated priors.
\subsection{The Kormendy Relation}
The Kormendy relation essentially a projection of the fundamental
plane.  It states that for elliptical galaxies, 
$R_e \propto \langle I \rangle^{0.83}$.  
We take, as a prior on the bulge radius and flux, a
distribution that is normally distributed around the line in $\log I$
- $\log R$ space defined by the Kormendy relation.

\subsection{Orientation Prior}
Assuming that the disk is oriented randomly in 3D space, 
they take on a non-uniform distribution of inclination angles.  By
assuming a uniform distribution of 3D orientations, we arrive at a
 prior proportional to $\gamma = \sqrt{\gamma_1^2 + \gamma_2^2}$.
If $\phi$ is the inclination angle ($\phi = 0$ for a face-on disk), then for an ideal
(circular, perfectly thin) disk, the shear $\gamma$ is given by
\[
\gamma = \frac{a - b}{a + b} = \frac{1 - \cos(\phi)}{1 + \cos(\phi)}
\Rightarrow \phi = \arccos{\left( \frac{1 - \gamma}{1 + \gamma} \right) }
\]
where $a$ and $b$ are the semi-major and semi-minor axes, respectively.

The prior pdf on $\gamma$ is 
\[
p(\gamma) = p(\phi) \frac{\partial \phi}{\partial \gamma}
\]
where $p(\phi)$ is the prior pdf on $\phi$. It can be shown that
$p(\phi) \propto \sin (\phi)$.  We can write $p(\phi)$ as
\[
p(\phi) \propto \sin(\phi)
= \sin \left( \arccos \left( \frac{1 + \gamma}{1 - \gamma} \right) \right)
= \sqrt{1 - \left( \frac{1 - \gamma}{1 + \gamma} \right)^2}
= \frac{2 \sqrt{\gamma}}{1 + \gamma}
\]
And we can write $\frac{\partial \phi}{\partial \gamma}$ as
\[
\frac{\partial \phi}{\partial \gamma} 
= \frac{\sqrt{\frac{\gamma}{(1 + \gamma)^2}}}{\gamma}
= \frac{1}{\sqrt{\gamma} (1 + \gamma)}
\]
Thus, 
\[
p(\gamma) \propto \frac{2 \sqrt{\gamma}}{1 + \gamma} \frac{1}{\sqrt{\gamma} (1 + \gamma)}
\propto \frac{1}{(1 + \gamma)^2}
\]


\section{Doing fits}
At first, we tried using a simple custom quadratic MLM estimator
\comment{What's the technical terminology?}.  Due to difficulties in
fitting parameters with complex constraints, we switched to MCMC
fitting with \href{http://dan.iel.fm/emcee/current/}{emcee}.  The
chain is run with flat priors, then the ones mentioned above are
applied after the chain has finished with importance sampling.

\subsection{Convergence tests}
We shoot for a burn-in time a few times larger the longest
autocorrelation time out of each parameter, and we run the chain for
about an order of magnitude longer than the longest autocorrelation
time.

Ideally, we would like an acceptance fraction of ~50\% for each chain, but 
we often get closer to ~30\%.

\subsection{Signal to Noise Ratio}
Simulated images are generated at a user-specified SNR.  This is
acomplished with \code{galsim.noise.addNoiseSNR}, which uses the
definition of SNR from Great08.  See Galsim
\href{http://galsim-developers.github.io/GalSim/namespacegalsim_1_1noise.html#a86f91278331ac9c134c25b16d4c67a32}{emcee}
for more info.

\end{document}
